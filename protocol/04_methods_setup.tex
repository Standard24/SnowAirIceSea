%---Linda

\subsection{Methods}

Regarding to results of the previous years, it is given that the glacier velocity for the glaciers Tellbreen and Blekumbreen is less then one meter.
The general used GPS system has only an accurancy in the order of meters.
By correcting the raw data with the data from the base station, it is posssible to have a more accurate positioning to the order of millimeters.
Carrier-based system, carrier-phase measurements ???
%(\cite{UGPS})
The accurancy depends on the distance to the base station (Gölles, 2012) and differs between the horizontal and the vertical component.
The accurancy can reached because the coordinates of the base station ar well known. 
So the disturbances by the atmosphere during the measuring time can be corrected, while the measured coordinates are compare on every timestep defined by the GPS time.. 
The further steps of the post processing is explanined in the next section 
%\ref{\section{Processing}}
We also did two measurements at the same stake on two different days to see how the results are effected.
\medskip

The GPS measurements have been done with the Trimble differential Global Navigation Satellite System (GNSS). 
The measured parameters were the northing and easting as well as the elevation.
The setting during our measurements inculded the receiver Trimble R4, the controller Trimble TCS2 and a carbon pole to mount the receiver proberly next to the stake (see figure).\medskip

During the operation of the measuremts we followed exact the description in section 2 and 3 in Gölles (2012). 
The recommended Fast Static survey method is used.
The used coordinate system is the Universal Transversal Mercator (UTM) for the zone 33x with WGS 1984 date. 
The duration for our measurments with the GPS receiver was at least 15 minutes. 
We had to made choice with a trade off between the quality of the result and the total number of measurements to measure all stakes at least one time. 
With a longer measurement time it is possible to average out the fluctuations better.


\subsection{Setup}

The setup for our measurments is specifed by a specific guidelines to insure that the measurements are consistent during the whole fieldwork.
The rover on the carbon pole has to be positioned on the top on the stake as far as possible. 
To determine the error from a tilted stake, it is necessary to measure the inclination of the stake as well as the direction of inclination with the compass.
The snow depth is measured with the probe which has a centimeter scale on it.
With snow depth and the antenna height the actual height over the ice surface is known. 
When the stake is already melted out too high, it is necessary to put the pole with the rover next to the stake. 
In this case the poles has to be positioned northwards of the stake so that the correction is only to the northing component, which makes the calculation easier. 
Also the distance from rover pole to stake is measured.
These two setups are shown in photographs in figure \ref{GPS:fig:setup}.

\begin{figure}
\centering
\includegraphics[width=0.48\linewidth]{./figs/pictures/GPS_setup.jpg}
\includegraphics[width=0.485\linewidth]{./figs/pictures/setup_ontop.JPG}
\caption{Setup while the GPS measurement next to the stake (left) and on top of the stake (right).}
\label{GPS:fig:setup}
\end{figure}
