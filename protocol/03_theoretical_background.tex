%\usepackage[parfill]{parskip}   
%
%\usepackage{bm}

%\section{}
%\subsection{}
%\chapter{}

%\begin{equation}...\end{equation}

%\begin{center}
%\end{center}

%\begin{itemize}
%\item blah blah
%\end{itemize}

%\begin{description}
%\item blah blah
%\end{description}

%\begin{enumerate}
%\item blah blah
%\end{enumerate}

%\begin{flushright}
%\end{flushright}

%\begin{figure}
%	\begin{center}
%		\includegraphics[scale=0.7]{nomdupdf}
%		\caption{Graphique}
%	\end{center}
%\end{figure}

%\listoffigures
%\listoftables

%\label{etiquette}
%\ref{etiquette}
%\pageref{etiquette}

%\begin{align*}
%A&=B\\
%&=C\\
%&=D
%\end{align*}

%\fbox


%\newcommand{\fraction}[2]{\raisebox{0.5ex}{#1} \slash \raisebox{-0.5ex}{#2}}


\subsection*{Ice velocities using the stake positions of the previous year}

If we define each measured position of the mass balance stakes by $P_i(x,y,z)$ where $P$ accounts for the codename of the stake and $i$ labels the year of the considered stake, the horizontal distance $d$ between a stake $P_a(x_a, y_a, z_a)$ and $P_b(x_b, y_b, z_b)$ is given by :

\begin{equation}d = \sqrt{(x_a - x_b)^2 + (y_a - y_b)^2}\end{equation}

The horizontal surface velocity $u_s$ is calculated through :

\begin{equation}
\boxed{u_s = \frac{d}{t_{a \rightarrow b}}}
\end{equation}

where $t_{a \rightarrow b}$ is the elapsed time between the measurements $a$ and $b$.


\subsection*{Relating the surface velocity to depth-averaged ice velocity}

Now that we have the horizontal surface velocity, we wonder how it evolves with depth. \cite{Nye1952} has proven that the overlying ice of a glacier must move at least as fast as that below. In quantitative terms, this translates to a reasoning that starts with the flow law for the strain rate $\dot{\epsilon}$ :

\begin{equation}\dot{\epsilon} = \left( \frac{\sigma}{B} \right)^n\end{equation}

with $\sigma$ a dominant shear stress, $B$ an ice viscosity parameter that is temperature dependent and increases with the stiffness of the ice, and $n$ the constant creep exponent. The latter has often experimentally proven to be $n \approx 3$, but for the sake of the derivation, we'll keep the value variable.
This law is also named after \cite{Glen1955}, who conducted the first uniaxial compression experiments on ice.

We will follow up with the assumptions of the layer we consider being parallel-sided, that there's no flow in $y$ direction and that everything is uniform in $x$, $y$, as depicted in the Figure \ref{velocities}. The stress rate tensor therefore comes down to :


\begin{equation} \dot{\epsilon} = \left( \begin{array}{ccc}
0 & 0 & \dot{\epsilon}_{xz} \\
0 & 0 & 0 \\
\dot{\epsilon}_{zx} & 0 & 0
\end{array} \right) \end{equation}

From the constitutive relations, linking the stress $\sigma$ to the strain $\epsilon$ component per component, it follows that $\sigma_{xz} = \sigma_{zx}$ and hence :

\begin{equation}\dot{\epsilon}_{zx} = \left( \frac{\sigma_{zx}}{B} \right)^n\end{equation}


\begin{figure}
	\begin{center}
		\includegraphics[scale=0.7]{VelocityProfile}
		\caption{Schematic diagram of our velocity profile}
		\label{velocities}
	\end{center}
\end{figure}

By the property $\dot{\epsilon}_{xy} = \frac{1}{2} \left( \frac{\partial u}{\partial y} + \frac{\partial v}{\partial x}\right)$, assuming the shear to take place in the plane normal to $z$ so that $\partial w / \partial x = 0$ and the stress configuration to be of simple shear, the last equation becomes :

\begin{equation}\frac{\mathrm{d} u}{\mathrm{d} z} = 2 \left( \frac{\sigma_{zx}}{B}\right)^n\end{equation}

We can express $\sigma_{zx}$ as a function of $z$ by using the coordinate system of Figure \ref{velocities} and noticing that $\sigma_{zx} = -\rho g h \sin{\alpha}$. After that, integration from the surface to the depth $z$ becomes possible :

\begin{equation}\int_{u_s}^{u(z)} \mathrm{d} u = -2 \left( \frac{\rho g \sin{\alpha}}{B}\right)^n \cdot \int_{0}^{z} z^n \mathrm{d} z\end{equation}


The depth-averaged ice velocity appears after carrying out the integration and rearranging the terms as conducted in \cite{Hooke2005} :

\begin{equation}u(z) = u_s - \frac{2}{n + 1} \left( \frac{\rho g \sin{\alpha}}{B}\right)^n \cdot z^{n+1}\end{equation}

This value becomes computable as soon as we know $u_s$, $B$ and $\alpha$. If we also know the total thickness $H$, the last equation becomes solvable at the bed and we get the bedrock velocity $u_b$ :

\begin{equation}u_b = u_s - \frac{2}{n + 1} \left( \frac{\rho g \sin{\alpha}}{B}\right)^n \cdot H^{n+1}\end{equation}

Our derivation here only is rigorously correct for a slab-shaped glacier of an infinite extent on a uniform slope. If for instance the glacier is bounded laterally, we will have to consider drag on the sides when calculating $\sigma_{zx}$ :

\begin{equation}\sigma_{zx} = -S_f \rho g z \sin{\alpha}\end{equation}

where we have introduced the shape factor $S_f$ ; we possess tables of values for different shapes, but it's good to remember that $S_f$ is $1$ for an infinitely wide glacier, and $\fraction{1}{2}$ for a semicircular glacier.

With that addition, our depth-averaged ice velocity finally reads :

\begin{equation}
\boxed{u(z) = u_s - \frac{2}{n + 1} \left( \frac{S_f \rho g \sin{\alpha}}{B}\right)^n \cdot z^{n+1}}
\end{equation}


\subsection*{The \textit{shallow ice} approximation and theory-motivated velocity}

This approximation concerns large ice sheets where conditions generally vary little over horizontal distances of 5-10 times the local ice thickness \cite{Greve2009}. Ice flow then becomes determined by local conditions such as ice thickness, surface inclination and temperature.
In addition to this, we add some further assumptions :

\begin{itemize}
    \item our bedrock shall be \textit{flat} throughout our measurements
    \item the bedrock shall be frozen, and hence : $u_b = 0$
\end{itemize}

With this basis, the previously derived expression for the base velocity $u_b$ becomes :

\begin{equation}u_b = 0 = u_\circledS - \frac{2}{n + 1} \left( \frac{\rho g \sin{\alpha}}{B}\right)^n \cdot H^{n+1}
\quad
\stackrel{n = 3}{\Longrightarrow}
\quad
\boxed{u_\circledS = \frac{1}{2} \left( \frac{\rho g \sin{\alpha}}{B}\right)^3 \cdot H^4}
\label{GPS:eq:sia}
\end{equation}

where $u_\circledS$ uses the circled subscript for the theory-motivated value of the ice velocity. We will keep using $u_s$ for the sake of generalization, but our theoretical computations in the next chapter will invoke this newly presented $u_\circledS$. 


\subsection*{Calculating the ice flux through an arbitrary cross section profile}

From definition, the ice flux per unit width $q$ is obtained by integrating the velocity profile over depth :

\begin{equation}q = \int_0^H u(z) \, \mathrm{d} z = u_s H - \frac{2}{(n+1)(n+2)} \left( \frac{S_f \rho g \sin{\alpha}}{B}\right)^n \cdot H^{n+2}\end{equation}

Introducing at this stage the mean velocity over depth $\bar{u} \equiv q/H$ reveals to be of greater advantage :

\begin{equation}\bar{u} = u_s - \frac{2}{(n+1)(n+2)} \left( \frac{S_f \rho g \sin{\alpha}}{B}\right)^n \cdot H^{n+1}\end{equation}

From here, after some calculus appears the very ergonomic expression :

\begin{equation}\bar{u} = \frac{4}{5}u_s + \frac{1}{5}u_b = \frac{4}{5}u_\circledS\end{equation}

This last expression allows us to jump back to a formula for the ice flux, reversing the definition of the mean velocity :

\begin{equation}
\boxed{q = \bar{u}H = \frac{4}{5}u_\circledS H}
\end{equation}


\subsection*{Mass balance and the \textit{steady state} assumption}

The massive ice flux per unit width can be defined through :

\begin{equation}q_m = \bar{\rho}q\end{equation}

where $\bar{\rho}$ is the vertically averaged density.

If we define an arbitrary planar cross-section $A$ of width $W$ around the positions where we measured $u_s$, and introduce the mass $M$ of the volume area comprised by $A \cdot H$, then the rate at which it changes would usually be defined \cite{Cuffey2010} by :
 
\begin{equation} \frac{\mathrm{d} M}{\mathrm{d}t} \equiv \bar{\rho} \int_A \dot{b}_i \, \mathrm{d}A - \int_W q_m \, \mathrm{d} W \end{equation}

where $\dot{b}_i$ is the ice-equivalent specific mass balance rate in $\left[ \fraction{m}{yr} \right]$. If we know the direction of the flow, it then becomes easier to pick a arch of width $W$ across the glacier into which $A$ would flow. We shall discover later that this is particularly difficult on both Tellbreen and Blekumbreen, partly because of the small order or magnitude of the measured velocities, which is in accordance with our further theoretical computations, but also because of the quite incoherent positioning over the years of the moving stakes, which makes it impossible to pick a favorite direction for the flux.

Yet if we decide to set the velocity to be zero as a prolongation of our experimental observation and predictions from the theory, then we'd have :

\begin{equation} \boxed{\frac{\mathrm{d} M}{\mathrm{d}t} \stackrel{u_s = 0}{=} \bar{\rho} \int_A \dot{b}_i \, \mathrm{d}A} \end{equation}

In the first order of approximation, should the right-hand side be anything else than zero, this equality would allow us to deny the \textit{steady state} assumption as any non-zero value to be multiplied by $\bar{\rho}$ and integrated over an area $A$ would mean a mass changing rate in the glacier.