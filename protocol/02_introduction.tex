Many topics in glaciology call for an understanding of the velocity field in a glacier. 
In fact, the way in which the flow redistributes mass is directly connected to a glacier's shape. But flow also induces energy redistribution and as such, the temperature distribution, which itself has implications for the nature of the coupling with the glacier bed. 
Spatial variations in speed, also known as strain rates $\epsilon$, are being studied by structural geologists who use glaciers as analogs for rock deformation. 
From a geomorphic point of view as well, the entrainment of debris and the different moraines they construct depend upon the flow field too. 
Understanding those velocity fields therefore is fundamental to the analysis of a number of problems in glacier mechanics.
In this experiments aiming at ice velocities and fluxes, we used the differential Global Positioning System (dGPS) to collect accurate data at the mass balance stakes' postions. Relying on these measurements to extract experimental values for the surface velocity of the glacier, we compared previous positions from the last years to the new ones.
We also particularly dedicated our attention towards the data treatment algorithms, as a clear understanding of the processing appears to be fundamental in an experiment with very sensitive devices and potential for large latent errors. In that scope, we introduced a new method based on an open-source post-processing that allows a transparent step-by-step understanding of the work-flow.
For consistency however, we kept conducting the Trimble Business Center post-processing next to our new design, be it only to ensure continuity with the previous years' results.
We focused on the measurement method because the results from the previous years seemed to be rather unclear or unreliable, and could hence not provide a real movement of the glacier. We therefore thought important to emphasize on the analysis of the measurement methods and their uncertainties, and herewith suggest a more rigorous protocol aiming at a more durable and reliable data collection on the field in the upcoming editions of this experiment.
\medskip 

The chapter on Theory walks us through the calculation of the horizontal velocity in a glacier. After having motivated the computation of the surface velocity through the \textit{shallow ice} approximation, we relate it to a depth-averaged velocity. Using this result, we present a value for the ice flux per unit width. Finally, we cross-check our results with those of the mass-balance group and draw a conclusion rejecting the \textit{steady state} assumption.
In the Processing chapter, the stake's effective positions are determined by the post-processing relying on base-station data and the stake's corrections. The latter is motivated by measurements of inclination and height that revealed the importance and necessity of considering the propagation of uncertainties.
With the position data including Northing, Easting and elevation, the velocity of every stake can be calculated both through the theory and by taking the difference between the stakes' positions of the last years. It is here that support provided by the data of the ice-radar group is determining ; we use the ice thickness that they extrapolated from their glacier scanning to provide a comparative theoretical value to the experimental surface velocity.
Next to the horizontal velocity, it was possible to calculate the vertical velocity to determine a mass balance for our measurement sites. 
Finally, the discussion and conclusion close the report.