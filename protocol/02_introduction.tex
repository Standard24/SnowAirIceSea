\documentclass[11pt]{report}
\usepackage{geometry}
\usepackage[parfill]{parskip}   
\usepackage{graphicx}
\usepackage{amssymb}
\usepackage{cite}
\usepackage{amsmath}
\usepackage{bm}
\usepackage{epstopdf}
\usepackage[latin1]{inputenc}
\usepackage[T1]{fontenc}
\usepackage{url}
\usepackage{lscape}
\usepackage{lastpage}
\usepackage{array}
\usepackage{siunitx}
\usepackage{verbatim}
\usepackage{fancyhdr}
\usepackage{textcomp}
\geometry{a4paper, body={180mm,257mm}, left=15mm,top=20mm, headheight=7mm,headsep=4mm, marginparsep=4mm, marginparwidth=17mm}
\pagestyle{fancy}

\DeclareGraphicsRule{.tif}{png}{.png}{`convert #1 `dirname #1`/`basename #1 .tif`.png}

\begin{document}

\newcommand{\fraction}[2]{\raisebox{0.5ex}{#1} \slash \raisebox{-0.5ex}{#2}}

Many topics in glaciology call for an understanding of the velocity field in a glacier. In fact, the way in which the flow redistributes mass is directly connected to a glacier's shape. But flow also induces energy redistribution and as such, the temperature distribution, which itself has implications for the nature of the coupling with the glacier bed. Spatial variations in speed, also known as strain rates $\epsilon$, are being studied by structural geologists who use glaciers as analogs for rock deformation. From a geomorphic point of view as well, the entrainment of debris and the different moraines they construct depend upon the flow field too. Understanding those velocity fields therefore is fundamental to the analysis of a number of problems in glacier mechanics.


The chapter on theory walks us through the calculation of the horizontal velocity in a glacier. After deriving the depth-averaged velocity, we use the \textit{shallow ice approximation} to compute a theory-motivated surface velocity. Using this result, we make an assumption on the glacier's cross-section around each stake we probed, and present a value for the ice flux of each glacier's delimited part. Finally, we cross-check our results with those of the mass-balance group and draw a conclusion relating to the \textit{steady state assumption}.

\end{document}