Many topics in glaciology call for an understanding of the velocity field in a glacier. 
In fact, the way in which the flow redistributes mass is directly connected to a glacier's shape. But flow also induces energy redistribution and as such, the temperature distribution, which itself has implications for the nature of the coupling with the glacier bed. 
Spatial variations in speed, also known as strain rates $\epsilon$, are being studied by structural geologists who use glaciers as analogs for rock deformation. 
From a geomorphic point of view as well, the entrainment of debris and the different moraines they construct depend upon the flow field too. 
Understanding those velocity fields therefore is fundamental to the analysis of a number of problems in glacier mechanics.
We used the differential Global Positioning Sytem (dGPS) to get accurate data on the mass balance stakes' postions.
We rely on these measurements to extract experimental values for the surface velocity of the glacier, comparing previous positions from the last years to the new ones.
We also address the data treatement algorithm as we strive for a clear understanding of the processing.
In that scope we introduce, with the open source post processing, a new method that allows a transparent step by step understanding of the work flow.
For consistency howerver we also conducted the Trimble Business Center post processing to asure continuity with the previous years' results. 
We focused on the measurement method because the results from the last years were unclear and could not provide a real movement of the glacier.
We therefore thought important to emphasize on the analysis of the measurement methods and their uncertainties, and herewith suggest a more rigourous protocol aiming at a more durable and reliable data collection on the field.
\medskip 

The chapter on theory walks us through the calculation of the horizontal velocity in a glacier. After having motivated the computation of the surface velocity through the \textit{shallow ice approximation}, we relate it to a depth-averaged velocity. Using this result, we make an assumption on the glacier's cross-section around each stake we probed and present a value for the ice flux of each glacier's delimited part. Finally, we cross-check our results with those of the mass-balance group and draw a conclusion relating to the \textit{steady state assumption}.
In the processing part the final position determined by the post processing with base station data and the stake correction caused by uncertainties in our setup in the field.
Based on this, it is important to consider the propagation of uncertainties.
With the position data, including northing, easting and elevation, the velocity of every stake can be calculated by the differnce of the stake positions of the last years.
Next to the horizontal velocity, it was possible to calculate the vertical velocity to determine a mass balance for our measurement sites. 
Finally, the discussion and conclusion close the report.
