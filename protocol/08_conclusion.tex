The new open source post processing method provides similar results for the stake positions.
Therewith the licensed TBC sofware can be replaced by the open source method.
Based on the results of the measured and theoretically calculated ice velocities we can confirm the low velocities measured the last years.
Due to a lot uncertainties in the measurement setup the uncertaint can only hardly reduced.
The uncertainties with 0.40 m for northing and 0.19 m for easting are the upper limit for the velocities. 
This leads to an high error in the velocity so the horizontal velocity not measurable.
But because of a faster movement in the vertical direction it is possible to calculate even with an uncertainty of 0.89 m, a vertical velocity.
Thus we are able to calulate a mass balance for both glacier, which shows us that the whole area of the glacier is in the ablation zone.
The theoretical surface velocity out of the shallow ice approximation validate the small values for the horizontal velocity.
Prospective, the measurement setup should be improved. 
Since, we tried the best of documentation and correction of the measurement data it is even not possible to reduce the uncertainties of the positioning and velocity significantly. 
The suggestion for the improvement is to drill the stakes defenetly straight to avoid a too high inclination.
Also the rover should be placed every time on top of the stake.
If the stake is to high melted out, it should be cut. 
This has to be documented very well too- 
Open source software can be used instead of TBC



uncertainties of tab 1.1 is upper limit for velocity

Horizontal velocity not measurable, better vertical.



Improvements...

-> put stakes straight
-> put rover in stake, cut stake
-> measure more stakes two times

