The aim of this report is to measure the ice velocity on the glaciers of Tellbreen and Blekumbreen (Adventdalen, Svalbard) with the differential Global Positioning System (dGPS) method. This method with a higher accuracy than the Global Positioning System (GPS) was necessary to address behaviours of very small velocities.
A new open-source post-processing method is applied and suggestions of long-term and convincing replacement of the former method are provided to slowly transition away from the restricted Trimble Business Center post-processing work-flow.
The average uncertainties of our calculated velocities are 0.40 [m] for Northing, 0.19 [m] for Easting, 0.89 [m] for the elevation, and are as such are larger than the calculated velocities themselves, making it impossible to determine an ice velocity for any stake on either of the two glaciers. This conclusion induces an irrelevancy to perform further calculation on the ice fluxes and their directions.
The mass balance of the glacier is calculated through the change of elevation in time at each stake. 
The mass balance gradient is of 5.1 $\pm$ 1.0 [mm/m] on Tellbreen and of 6.8 $\pm$ 1.4 [mm/m] on Blekumbreen. The altitudes of the theoretical equilibrium lines at 757 $\pm$ 56 [m] for Tellbreen and 751 $\pm$ 45 [m] for Blekumbreen are higher than the highest point of any of the two glaciers. Both glaciers are situated in their own ablation zone.
We verify our measurements with the theoretical model of the \textit{shallow ice} approximation. The theoretical velocity values have orders of magnitude of centimeters per year for reasonable assumptions of slope angles, ice viscosities and ice thicknesses. Their results confirm our measurement results with a slow movement of both glaciers.