The aim was to measure the ice velocity on the glaciers Tellbreen and Blekumbreen  with the differential Global Positioning System method. 
This method with a higher accurancy than the Global Positioning System method was necessary, because we try to resolve small velocities.
On the measurements a new open source post processing method is applied and is able to replace the Trimble Business Center post processing.
The average uncertainties of our calculated velocities are 0.40 m for northing, 0.19 m for easting and 0.89 m for the elevation.
This values are bigger than the calculated velocities, except some unrealistic high values, which means that we cannot determine a ice velocity for any stake both glaciers.
Therefore, no further calculation on the ice flux and their direction can made.
The mass balance of the glacier is calulated by the change of elevation in time at each stake. 
The mass balance gradient is on Tellbreen 5.1 $\pm$ 1.0 mm/m and on Blekumbreen 6.8 $\pm$ 1.4 mm/m. The elevation of theoretical equilibrium line with 757 $\pm$ 56 m for Tellbreen and 751 $\pm$ 45 m for Blekumbreen are higher than the highest point of the glacier. 
So both glacier are completly in the ablation zone.
To verify our measurements with the theoretical model of the shallow ice approximation.
The theoretical velocity values are in the order of magnitude of centimeters per year for reasonable assumption of slope angle, ice viscosity and ice thickness.
This result confirm our measurement results with a slow movement of both glaciers.