
\subsection{Horizontal velocity}

As the two glaciers move down the slope of the mountains,
we expect an eastward stake movement on Tellbreen and
a westward movement on Blekumbreen, with not too much variation of the velocity over the years.
For many of the stakes, this is not the case.
The movement is sometimes perpendicular to the flow line of the glaciers
and at times even uphill.
In the past years, this has been explained by a 'local squeeze and stretch movement' \citep{rep2016} or
the 'shrinking and swelling of a melt water channel' \citep{rep2017}.
Although those explanations may be true, we see no clear evidence of a horizontal glacier movement in the measurement data.
From the 18 stake velocities $v_{2017}$ which were calculated (Tab.~\ref{GPS:tab:vel_tab}),
eleven velocities include a velocity of zero within their uncertainty.
Some of the seven remaining velocities which are far from zero could be explained with outliers due to unknown reasons, e.g.
issues with the stake position of last year or a failure of the post processing algorithm.

In any case, a Gaussian distribution of the measured values implies that 32 \% of the values
do not include zero within their 1-$\sigma$ interval, if the actual glacier velocity is zero.
This corresponds to six out of 18 stakes.
Looking at the velocity values in Tab.~\ref{GPS:tab:vel_tab} this way, we can not reject the null hypothesis that the glaciers are not moving.

Nevertheless, a few stakes show some coherent movement.
Therefore each of the stakes will subsequently shortly be discussed.

\paragraph{T1}
The positions from 2009 until 2015 exhibit an intermittent eastward movement,
which is in accordance to the flow line direction at this position (east to south-east).
In 2015, the stake has moved about 2 m up the slope, which is implausible.
The values of 2016 and 2017 coincide within their uncertainties and show no movement at all.
2018 the measurement of the position has been performed twice.
The two positions agree to each other.
The two new data points are north of the 2017 position,
but considering the error bars, this direction is not significant.

\paragraph{T2}
The measurements before 2017 show no stake movement.
The two measurements of T2-2017 that have been conducted this year agree within their uncertainties and
also not significantly away from the position of 2017.
The older stake T2-2016 also did not move significantly.

\paragraph{T3}
None of the four stakes show any significant movement.

\paragraph{T4}
The measurements before 2015 show a slight northwards trend.
This years position of T4-2016 lies between the positions of 2016 and 2017.

\paragraph{T5}
The measurements before 2015 may have an intermittent north-eastwards trend.
T5-2016 is has not been moving significantly for the last three years.

\paragraph{T6}
Data from before 2015 are partly missing and this data show an unexpected jump of about 6~m southwards.
The 2016 position of T6-2016 lies between the values of 2017 and 2018,
but all three differ not significantly from each other.

\paragraph{T7}
Positions before 2015 are inconsistent.
The positions of T7-2015 lie all within their uncertainties since 2015.
T7-2017 exhibits an unexpected movement to north west.

\paragraph{T8}
For three out of four different stakes on location T8, all position measurements are indistinguishable due to uncertainty.
The two measurements of T8-2015 differ slightly more, but not very significantly in the expected eastwards direction.
\bigskip

\paragraph{BL2}
BL2-2011 shows a rather consistent westwards movement along the flow line of the glacier until 2015.
The subsequently placed stake BL2-2016 continues this tendency until 2018.
Therefore we can have some confidence in the calculated velocities
\mbox{$v_{2017} = 26 \pm 21\,$cm/a} and \mbox{$v_{2016} = 21 \pm 19\,$cm/a}.

\paragraph{BL3}
Also for BL3-2011, a westwards drift is visible.
The measurement value of 2013 is obviously an outlier.
BL3-2015 continues a north-west drift from 2015 until 2017.
The 2018 value for the position of BL3-2016 does not continue the trend of the prior stakes.
The positions of this stake from 2016 to 2018 are very close together.

\paragraph{BL4}
Likewise for BL4-2011, a slight westwards drift with an outlier in 2013 is observed.
The more recent position measurements of BL4-2016 are not distinguishable.

\paragraph{BL5}
BL5-2011 has constantly been moving westwards for seven years, from 2011 until 2017.
Surprisingly, this years position of BL5-2017 suggests a velocity as large as
$1.75 \pm 0.18\,$m/a and $1.74 \pm 0.24\,$m/a (Tab.~\ref{GPS:tab:vel_tab}).
As the position measurement has been conducted two times in 2018 and the obtained positions are almost the same,
we suspect an error in the position measurement of 2017.
\bigskip

In summary, the analysis of the entire data set of the stake positions on Tellbreen seems to provide
no evidence of a consistent glacier movement.
Single events of stake movements in various directions
could be explained by local and short-time displacements due to the forming
of melt water channels or crevasses, like in the last reports.
But looking at the bigger picture of all available data,
we advance the view that a movement of Tellbreen is not detectable with the applied measurement method,
and that individual apparent stake movements are caused by the uncertainty of the method.
This view is supported by a comparison with the theoretical values for the glacier velocity.
As the measurements provide no reliable value for the glacier velocity,
a reasonable calculation of the ice flow volume can not be made.

On Blekumbreen, the situation is different.
The investigation of the seven year long time series since 2011 makes a trend visible,
which could hardly be discovered by comparing positions of only two consecutive years.
Even though the uncertainties make it impossible to show a movement of Blekumbreen with only this year's
and last year's positions,
looking back longer in the past gives clear evidence of a small, but coherent movement along the flow line of Blekumbreen.

To improve the quality of the results,
the main sources of uncertainty have to be identified.
One large source is the GPS measurement itself.
The surrounding mountains are covering a great part of the sky,
thereby hiding a few satellites and probably prohibit a fixed solution in some measurements.
Without a fixed solution the result can be used, but cannot be expected to be an exact positioning.
Reflections from the metal stake could deteriorate the signals and should be avoided.
The distance to the base station on Platåbreen is relatively long and the time series of some measurements show,
that the correction with the base station data can not fully remove artefacts,
possibly due to atmospheric disturbances.\\
The second big error source is the inclination of some stakes.
In conjunction with ablation, the inclination of the stake adds a displacement which has to be corrected.
The therefore necessary measurement of angle and direction of inclination can only be performed with a high uncertainty.
For example, a stake inclination of 10\,$^\text{o}$ together with an ablation of 1\,m shifts the entry point of the stake
into the ice by 17\,cm.
If this displacement is not corrected,
it adds an error in the order of the annual displacement of the glacier.\\
Another source of error is that each years group employs slightly different proceedings in terms
of the measurement setup and the data post processing.

An interesting result of the open source post processing is that the mean uncertainty on the northing is
more than twice as large as on the easting (Tab.~\ref{GPS:tab:errors}).
The reason for this could be the location of Svalbard close to the north pole,
which creates an asymmetric satellite geometry more suitable to measure easting than northing.
Also in most of the measurements the rover is positioned northwards, 
which gives an additional error source due to the uncertainty of the measured distance between rover and stake.

\subsection{Vertical velocity}

The analysis of the vertical stake movement yields much more reliable results than for the horizontal velocity.
Even though an analysis of the vertical component has already been performed twice
(\cite{rep2016} and \cite{rep2017}),
we are the first group who could calculate the equilibrium line elevation and the mass balance gradient for
both glaciers only from GPS data.
The results are in good agreement with the findings of the mass balance group, who report equilibrium lines of
673\,m for Tellbreen and 678\,m for Blekumbreen as well as
mass balance gradients of 5.7 mm w.e./m for Tellbreen and 5.3 mm w.e/m for Blekumbreen.\\
The individual mass balances of the stakes exhibit a linear behaviour for both glaciers.
Only T8 and T6 on Tellbreen have a higher ablation than expected from the linear fit.
This could be due to local variations of the terrain around the stakes,
like local radiation or wind conditions.

Both equilibrium lines and mass balance gradients for Tellbreen and Blekumbreen coincide within their uncertainties.
The equilibrium line elevations are about 100\,m above the highest stake on Tellbreen,
which means that there are no accumulation zones on the glaciers.

Even though measuring mass balance by GPS is a much coarser method than measuring it with a tape measure on stakes,
we think that by considering a long time series of elevation measurements,
the GPS method could eventually be even more precise,
because statistical fluctuations of single years have less influence.
Additionally, measurement errors of single years can easily be detected,
because elevation values of not consecutive years can be compared.
This is not possible with the usual method of measuring the mass balance with a ruler,
when new stakes have to be placed.

